
% Default to the notebook output style

    


% Inherit from the specified cell style.




    
\documentclass[11pt]{article}

    
    
    \usepackage[T1]{fontenc}
    % Nicer default font (+ math font) than Computer Modern for most use cases
    \usepackage{mathpazo}

    % Basic figure setup, for now with no caption control since it's done
    % automatically by Pandoc (which extracts ![](path) syntax from Markdown).
    \usepackage{graphicx}
    % We will generate all images so they have a width \maxwidth. This means
    % that they will get their normal width if they fit onto the page, but
    % are scaled down if they would overflow the margins.
    \makeatletter
    \def\maxwidth{\ifdim\Gin@nat@width>\linewidth\linewidth
    \else\Gin@nat@width\fi}
    \makeatother
    \let\Oldincludegraphics\includegraphics
    % Set max figure width to be 80% of text width, for now hardcoded.
    \renewcommand{\includegraphics}[1]{\Oldincludegraphics[width=.8\maxwidth]{#1}}
    % Ensure that by default, figures have no caption (until we provide a
    % proper Figure object with a Caption API and a way to capture that
    % in the conversion process - todo).
    \usepackage{caption}
    \DeclareCaptionLabelFormat{nolabel}{}
    \captionsetup{labelformat=nolabel}

    \usepackage{adjustbox} % Used to constrain images to a maximum size 
    \usepackage{xcolor} % Allow colors to be defined
    \usepackage{enumerate} % Needed for markdown enumerations to work
    \usepackage{geometry} % Used to adjust the document margins
    \usepackage{amsmath} % Equations
    \usepackage{amssymb} % Equations
    \usepackage{textcomp} % defines textquotesingle
    % Hack from http://tex.stackexchange.com/a/47451/13684:
    \AtBeginDocument{%
        \def\PYZsq{\textquotesingle}% Upright quotes in Pygmentized code
    }
    \usepackage{upquote} % Upright quotes for verbatim code
    \usepackage{eurosym} % defines \euro
    \usepackage[mathletters]{ucs} % Extended unicode (utf-8) support
    \usepackage[utf8x]{inputenc} % Allow utf-8 characters in the tex document
    \usepackage{fancyvrb} % verbatim replacement that allows latex
    \usepackage{grffile} % extends the file name processing of package graphics 
                         % to support a larger range 
    % The hyperref package gives us a pdf with properly built
    % internal navigation ('pdf bookmarks' for the table of contents,
    % internal cross-reference links, web links for URLs, etc.)
    \usepackage{hyperref}
    \usepackage{longtable} % longtable support required by pandoc >1.10
    \usepackage{booktabs}  % table support for pandoc > 1.12.2
    \usepackage[inline]{enumitem} % IRkernel/repr support (it uses the enumerate* environment)
    \usepackage[normalem]{ulem} % ulem is needed to support strikethroughs (\sout)
                                % normalem makes italics be italics, not underlines
    

    
    
    % Colors for the hyperref package
    \definecolor{urlcolor}{rgb}{0,.145,.698}
    \definecolor{linkcolor}{rgb}{.71,0.21,0.01}
    \definecolor{citecolor}{rgb}{.12,.54,.11}

    % ANSI colors
    \definecolor{ansi-black}{HTML}{3E424D}
    \definecolor{ansi-black-intense}{HTML}{282C36}
    \definecolor{ansi-red}{HTML}{E75C58}
    \definecolor{ansi-red-intense}{HTML}{B22B31}
    \definecolor{ansi-green}{HTML}{00A250}
    \definecolor{ansi-green-intense}{HTML}{007427}
    \definecolor{ansi-yellow}{HTML}{DDB62B}
    \definecolor{ansi-yellow-intense}{HTML}{B27D12}
    \definecolor{ansi-blue}{HTML}{208FFB}
    \definecolor{ansi-blue-intense}{HTML}{0065CA}
    \definecolor{ansi-magenta}{HTML}{D160C4}
    \definecolor{ansi-magenta-intense}{HTML}{A03196}
    \definecolor{ansi-cyan}{HTML}{60C6C8}
    \definecolor{ansi-cyan-intense}{HTML}{258F8F}
    \definecolor{ansi-white}{HTML}{C5C1B4}
    \definecolor{ansi-white-intense}{HTML}{A1A6B2}

    % commands and environments needed by pandoc snippets
    % extracted from the output of `pandoc -s`
    \providecommand{\tightlist}{%
      \setlength{\itemsep}{0pt}\setlength{\parskip}{0pt}}
    \DefineVerbatimEnvironment{Highlighting}{Verbatim}{commandchars=\\\{\}}
    % Add ',fontsize=\small' for more characters per line
    \newenvironment{Shaded}{}{}
    \newcommand{\KeywordTok}[1]{\textcolor[rgb]{0.00,0.44,0.13}{\textbf{{#1}}}}
    \newcommand{\DataTypeTok}[1]{\textcolor[rgb]{0.56,0.13,0.00}{{#1}}}
    \newcommand{\DecValTok}[1]{\textcolor[rgb]{0.25,0.63,0.44}{{#1}}}
    \newcommand{\BaseNTok}[1]{\textcolor[rgb]{0.25,0.63,0.44}{{#1}}}
    \newcommand{\FloatTok}[1]{\textcolor[rgb]{0.25,0.63,0.44}{{#1}}}
    \newcommand{\CharTok}[1]{\textcolor[rgb]{0.25,0.44,0.63}{{#1}}}
    \newcommand{\StringTok}[1]{\textcolor[rgb]{0.25,0.44,0.63}{{#1}}}
    \newcommand{\CommentTok}[1]{\textcolor[rgb]{0.38,0.63,0.69}{\textit{{#1}}}}
    \newcommand{\OtherTok}[1]{\textcolor[rgb]{0.00,0.44,0.13}{{#1}}}
    \newcommand{\AlertTok}[1]{\textcolor[rgb]{1.00,0.00,0.00}{\textbf{{#1}}}}
    \newcommand{\FunctionTok}[1]{\textcolor[rgb]{0.02,0.16,0.49}{{#1}}}
    \newcommand{\RegionMarkerTok}[1]{{#1}}
    \newcommand{\ErrorTok}[1]{\textcolor[rgb]{1.00,0.00,0.00}{\textbf{{#1}}}}
    \newcommand{\NormalTok}[1]{{#1}}
    
    % Additional commands for more recent versions of Pandoc
    \newcommand{\ConstantTok}[1]{\textcolor[rgb]{0.53,0.00,0.00}{{#1}}}
    \newcommand{\SpecialCharTok}[1]{\textcolor[rgb]{0.25,0.44,0.63}{{#1}}}
    \newcommand{\VerbatimStringTok}[1]{\textcolor[rgb]{0.25,0.44,0.63}{{#1}}}
    \newcommand{\SpecialStringTok}[1]{\textcolor[rgb]{0.73,0.40,0.53}{{#1}}}
    \newcommand{\ImportTok}[1]{{#1}}
    \newcommand{\DocumentationTok}[1]{\textcolor[rgb]{0.73,0.13,0.13}{\textit{{#1}}}}
    \newcommand{\AnnotationTok}[1]{\textcolor[rgb]{0.38,0.63,0.69}{\textbf{\textit{{#1}}}}}
    \newcommand{\CommentVarTok}[1]{\textcolor[rgb]{0.38,0.63,0.69}{\textbf{\textit{{#1}}}}}
    \newcommand{\VariableTok}[1]{\textcolor[rgb]{0.10,0.09,0.49}{{#1}}}
    \newcommand{\ControlFlowTok}[1]{\textcolor[rgb]{0.00,0.44,0.13}{\textbf{{#1}}}}
    \newcommand{\OperatorTok}[1]{\textcolor[rgb]{0.40,0.40,0.40}{{#1}}}
    \newcommand{\BuiltInTok}[1]{{#1}}
    \newcommand{\ExtensionTok}[1]{{#1}}
    \newcommand{\PreprocessorTok}[1]{\textcolor[rgb]{0.74,0.48,0.00}{{#1}}}
    \newcommand{\AttributeTok}[1]{\textcolor[rgb]{0.49,0.56,0.16}{{#1}}}
    \newcommand{\InformationTok}[1]{\textcolor[rgb]{0.38,0.63,0.69}{\textbf{\textit{{#1}}}}}
    \newcommand{\WarningTok}[1]{\textcolor[rgb]{0.38,0.63,0.69}{\textbf{\textit{{#1}}}}}
    
    
    % Define a nice break command that doesn't care if a line doesn't already
    % exist.
    \def\br{\hspace*{\fill} \\* }
    % Math Jax compatability definitions
    \def\gt{>}
    \def\lt{<}
    % Document parameters
    \title{Project 1 - Modeling Whooping Crane Population vFinal}
    
    
    

    % Pygments definitions
    
\makeatletter
\def\PY@reset{\let\PY@it=\relax \let\PY@bf=\relax%
    \let\PY@ul=\relax \let\PY@tc=\relax%
    \let\PY@bc=\relax \let\PY@ff=\relax}
\def\PY@tok#1{\csname PY@tok@#1\endcsname}
\def\PY@toks#1+{\ifx\relax#1\empty\else%
    \PY@tok{#1}\expandafter\PY@toks\fi}
\def\PY@do#1{\PY@bc{\PY@tc{\PY@ul{%
    \PY@it{\PY@bf{\PY@ff{#1}}}}}}}
\def\PY#1#2{\PY@reset\PY@toks#1+\relax+\PY@do{#2}}

\expandafter\def\csname PY@tok@w\endcsname{\def\PY@tc##1{\textcolor[rgb]{0.73,0.73,0.73}{##1}}}
\expandafter\def\csname PY@tok@c\endcsname{\let\PY@it=\textit\def\PY@tc##1{\textcolor[rgb]{0.25,0.50,0.50}{##1}}}
\expandafter\def\csname PY@tok@cp\endcsname{\def\PY@tc##1{\textcolor[rgb]{0.74,0.48,0.00}{##1}}}
\expandafter\def\csname PY@tok@k\endcsname{\let\PY@bf=\textbf\def\PY@tc##1{\textcolor[rgb]{0.00,0.50,0.00}{##1}}}
\expandafter\def\csname PY@tok@kp\endcsname{\def\PY@tc##1{\textcolor[rgb]{0.00,0.50,0.00}{##1}}}
\expandafter\def\csname PY@tok@kt\endcsname{\def\PY@tc##1{\textcolor[rgb]{0.69,0.00,0.25}{##1}}}
\expandafter\def\csname PY@tok@o\endcsname{\def\PY@tc##1{\textcolor[rgb]{0.40,0.40,0.40}{##1}}}
\expandafter\def\csname PY@tok@ow\endcsname{\let\PY@bf=\textbf\def\PY@tc##1{\textcolor[rgb]{0.67,0.13,1.00}{##1}}}
\expandafter\def\csname PY@tok@nb\endcsname{\def\PY@tc##1{\textcolor[rgb]{0.00,0.50,0.00}{##1}}}
\expandafter\def\csname PY@tok@nf\endcsname{\def\PY@tc##1{\textcolor[rgb]{0.00,0.00,1.00}{##1}}}
\expandafter\def\csname PY@tok@nc\endcsname{\let\PY@bf=\textbf\def\PY@tc##1{\textcolor[rgb]{0.00,0.00,1.00}{##1}}}
\expandafter\def\csname PY@tok@nn\endcsname{\let\PY@bf=\textbf\def\PY@tc##1{\textcolor[rgb]{0.00,0.00,1.00}{##1}}}
\expandafter\def\csname PY@tok@ne\endcsname{\let\PY@bf=\textbf\def\PY@tc##1{\textcolor[rgb]{0.82,0.25,0.23}{##1}}}
\expandafter\def\csname PY@tok@nv\endcsname{\def\PY@tc##1{\textcolor[rgb]{0.10,0.09,0.49}{##1}}}
\expandafter\def\csname PY@tok@no\endcsname{\def\PY@tc##1{\textcolor[rgb]{0.53,0.00,0.00}{##1}}}
\expandafter\def\csname PY@tok@nl\endcsname{\def\PY@tc##1{\textcolor[rgb]{0.63,0.63,0.00}{##1}}}
\expandafter\def\csname PY@tok@ni\endcsname{\let\PY@bf=\textbf\def\PY@tc##1{\textcolor[rgb]{0.60,0.60,0.60}{##1}}}
\expandafter\def\csname PY@tok@na\endcsname{\def\PY@tc##1{\textcolor[rgb]{0.49,0.56,0.16}{##1}}}
\expandafter\def\csname PY@tok@nt\endcsname{\let\PY@bf=\textbf\def\PY@tc##1{\textcolor[rgb]{0.00,0.50,0.00}{##1}}}
\expandafter\def\csname PY@tok@nd\endcsname{\def\PY@tc##1{\textcolor[rgb]{0.67,0.13,1.00}{##1}}}
\expandafter\def\csname PY@tok@s\endcsname{\def\PY@tc##1{\textcolor[rgb]{0.73,0.13,0.13}{##1}}}
\expandafter\def\csname PY@tok@sd\endcsname{\let\PY@it=\textit\def\PY@tc##1{\textcolor[rgb]{0.73,0.13,0.13}{##1}}}
\expandafter\def\csname PY@tok@si\endcsname{\let\PY@bf=\textbf\def\PY@tc##1{\textcolor[rgb]{0.73,0.40,0.53}{##1}}}
\expandafter\def\csname PY@tok@se\endcsname{\let\PY@bf=\textbf\def\PY@tc##1{\textcolor[rgb]{0.73,0.40,0.13}{##1}}}
\expandafter\def\csname PY@tok@sr\endcsname{\def\PY@tc##1{\textcolor[rgb]{0.73,0.40,0.53}{##1}}}
\expandafter\def\csname PY@tok@ss\endcsname{\def\PY@tc##1{\textcolor[rgb]{0.10,0.09,0.49}{##1}}}
\expandafter\def\csname PY@tok@sx\endcsname{\def\PY@tc##1{\textcolor[rgb]{0.00,0.50,0.00}{##1}}}
\expandafter\def\csname PY@tok@m\endcsname{\def\PY@tc##1{\textcolor[rgb]{0.40,0.40,0.40}{##1}}}
\expandafter\def\csname PY@tok@gh\endcsname{\let\PY@bf=\textbf\def\PY@tc##1{\textcolor[rgb]{0.00,0.00,0.50}{##1}}}
\expandafter\def\csname PY@tok@gu\endcsname{\let\PY@bf=\textbf\def\PY@tc##1{\textcolor[rgb]{0.50,0.00,0.50}{##1}}}
\expandafter\def\csname PY@tok@gd\endcsname{\def\PY@tc##1{\textcolor[rgb]{0.63,0.00,0.00}{##1}}}
\expandafter\def\csname PY@tok@gi\endcsname{\def\PY@tc##1{\textcolor[rgb]{0.00,0.63,0.00}{##1}}}
\expandafter\def\csname PY@tok@gr\endcsname{\def\PY@tc##1{\textcolor[rgb]{1.00,0.00,0.00}{##1}}}
\expandafter\def\csname PY@tok@ge\endcsname{\let\PY@it=\textit}
\expandafter\def\csname PY@tok@gs\endcsname{\let\PY@bf=\textbf}
\expandafter\def\csname PY@tok@gp\endcsname{\let\PY@bf=\textbf\def\PY@tc##1{\textcolor[rgb]{0.00,0.00,0.50}{##1}}}
\expandafter\def\csname PY@tok@go\endcsname{\def\PY@tc##1{\textcolor[rgb]{0.53,0.53,0.53}{##1}}}
\expandafter\def\csname PY@tok@gt\endcsname{\def\PY@tc##1{\textcolor[rgb]{0.00,0.27,0.87}{##1}}}
\expandafter\def\csname PY@tok@err\endcsname{\def\PY@bc##1{\setlength{\fboxsep}{0pt}\fcolorbox[rgb]{1.00,0.00,0.00}{1,1,1}{\strut ##1}}}
\expandafter\def\csname PY@tok@kc\endcsname{\let\PY@bf=\textbf\def\PY@tc##1{\textcolor[rgb]{0.00,0.50,0.00}{##1}}}
\expandafter\def\csname PY@tok@kd\endcsname{\let\PY@bf=\textbf\def\PY@tc##1{\textcolor[rgb]{0.00,0.50,0.00}{##1}}}
\expandafter\def\csname PY@tok@kn\endcsname{\let\PY@bf=\textbf\def\PY@tc##1{\textcolor[rgb]{0.00,0.50,0.00}{##1}}}
\expandafter\def\csname PY@tok@kr\endcsname{\let\PY@bf=\textbf\def\PY@tc##1{\textcolor[rgb]{0.00,0.50,0.00}{##1}}}
\expandafter\def\csname PY@tok@bp\endcsname{\def\PY@tc##1{\textcolor[rgb]{0.00,0.50,0.00}{##1}}}
\expandafter\def\csname PY@tok@fm\endcsname{\def\PY@tc##1{\textcolor[rgb]{0.00,0.00,1.00}{##1}}}
\expandafter\def\csname PY@tok@vc\endcsname{\def\PY@tc##1{\textcolor[rgb]{0.10,0.09,0.49}{##1}}}
\expandafter\def\csname PY@tok@vg\endcsname{\def\PY@tc##1{\textcolor[rgb]{0.10,0.09,0.49}{##1}}}
\expandafter\def\csname PY@tok@vi\endcsname{\def\PY@tc##1{\textcolor[rgb]{0.10,0.09,0.49}{##1}}}
\expandafter\def\csname PY@tok@vm\endcsname{\def\PY@tc##1{\textcolor[rgb]{0.10,0.09,0.49}{##1}}}
\expandafter\def\csname PY@tok@sa\endcsname{\def\PY@tc##1{\textcolor[rgb]{0.73,0.13,0.13}{##1}}}
\expandafter\def\csname PY@tok@sb\endcsname{\def\PY@tc##1{\textcolor[rgb]{0.73,0.13,0.13}{##1}}}
\expandafter\def\csname PY@tok@sc\endcsname{\def\PY@tc##1{\textcolor[rgb]{0.73,0.13,0.13}{##1}}}
\expandafter\def\csname PY@tok@dl\endcsname{\def\PY@tc##1{\textcolor[rgb]{0.73,0.13,0.13}{##1}}}
\expandafter\def\csname PY@tok@s2\endcsname{\def\PY@tc##1{\textcolor[rgb]{0.73,0.13,0.13}{##1}}}
\expandafter\def\csname PY@tok@sh\endcsname{\def\PY@tc##1{\textcolor[rgb]{0.73,0.13,0.13}{##1}}}
\expandafter\def\csname PY@tok@s1\endcsname{\def\PY@tc##1{\textcolor[rgb]{0.73,0.13,0.13}{##1}}}
\expandafter\def\csname PY@tok@mb\endcsname{\def\PY@tc##1{\textcolor[rgb]{0.40,0.40,0.40}{##1}}}
\expandafter\def\csname PY@tok@mf\endcsname{\def\PY@tc##1{\textcolor[rgb]{0.40,0.40,0.40}{##1}}}
\expandafter\def\csname PY@tok@mh\endcsname{\def\PY@tc##1{\textcolor[rgb]{0.40,0.40,0.40}{##1}}}
\expandafter\def\csname PY@tok@mi\endcsname{\def\PY@tc##1{\textcolor[rgb]{0.40,0.40,0.40}{##1}}}
\expandafter\def\csname PY@tok@il\endcsname{\def\PY@tc##1{\textcolor[rgb]{0.40,0.40,0.40}{##1}}}
\expandafter\def\csname PY@tok@mo\endcsname{\def\PY@tc##1{\textcolor[rgb]{0.40,0.40,0.40}{##1}}}
\expandafter\def\csname PY@tok@ch\endcsname{\let\PY@it=\textit\def\PY@tc##1{\textcolor[rgb]{0.25,0.50,0.50}{##1}}}
\expandafter\def\csname PY@tok@cm\endcsname{\let\PY@it=\textit\def\PY@tc##1{\textcolor[rgb]{0.25,0.50,0.50}{##1}}}
\expandafter\def\csname PY@tok@cpf\endcsname{\let\PY@it=\textit\def\PY@tc##1{\textcolor[rgb]{0.25,0.50,0.50}{##1}}}
\expandafter\def\csname PY@tok@c1\endcsname{\let\PY@it=\textit\def\PY@tc##1{\textcolor[rgb]{0.25,0.50,0.50}{##1}}}
\expandafter\def\csname PY@tok@cs\endcsname{\let\PY@it=\textit\def\PY@tc##1{\textcolor[rgb]{0.25,0.50,0.50}{##1}}}

\def\PYZbs{\char`\\}
\def\PYZus{\char`\_}
\def\PYZob{\char`\{}
\def\PYZcb{\char`\}}
\def\PYZca{\char`\^}
\def\PYZam{\char`\&}
\def\PYZlt{\char`\<}
\def\PYZgt{\char`\>}
\def\PYZsh{\char`\#}
\def\PYZpc{\char`\%}
\def\PYZdl{\char`\$}
\def\PYZhy{\char`\-}
\def\PYZsq{\char`\'}
\def\PYZdq{\char`\"}
\def\PYZti{\char`\~}
% for compatibility with earlier versions
\def\PYZat{@}
\def\PYZlb{[}
\def\PYZrb{]}
\makeatother


    % Exact colors from NB
    \definecolor{incolor}{rgb}{0.0, 0.0, 0.5}
    \definecolor{outcolor}{rgb}{0.545, 0.0, 0.0}



    
    % Prevent overflowing lines due to hard-to-break entities
    \sloppy 
    % Setup hyperref package
    \hypersetup{
      breaklinks=true,  % so long urls are correctly broken across lines
      colorlinks=true,
      urlcolor=urlcolor,
      linkcolor=linkcolor,
      citecolor=citecolor,
      }
    % Slightly bigger margins than the latex defaults
    
    \geometry{verbose,tmargin=1in,bmargin=1in,lmargin=1in,rmargin=1in}
    
    

    \begin{document}
    
    
    \maketitle
    
    

    
    \hypertarget{project-1---simulating-endangered-populations}{%
\section{Project 1 - Simulating Endangered
Populations}\label{project-1---simulating-endangered-populations}}

    Copyright 2018. Duncan Mazza and Aditya Sudhakar

*No model is correct, but some are useful

    \begin{Verbatim}[commandchars=\\\{\}]
{\color{incolor}In [{\color{incolor}26}]:} \PY{c+c1}{\PYZsh{} Configure Jupyter so figures appear in the notebook}
         \PY{o}{\PYZpc{}}\PY{k}{matplotlib} inline
         
         \PY{c+c1}{\PYZsh{} Configure Jupyter to display the assigned value after an assignment}
         \PY{o}{\PYZpc{}}\PY{k}{config} InteractiveShell.ast\PYZus{}node\PYZus{}interactivity=\PYZsq{}last\PYZus{}expr\PYZus{}or\PYZus{}assign\PYZsq{}
         
         \PY{c+c1}{\PYZsh{} import functions from the modsim library}
         \PY{k+kn}{from} \PY{n+nn}{modsim} \PY{k}{import} \PY{o}{*}
         \PY{k+kn}{from} \PY{n+nn}{pandas} \PY{k}{import} \PY{n}{read\PYZus{}html} 
\end{Verbatim}


    \hypertarget{can-we-predict-whether-an-endangered-species-will-pass-a-certain-population-threshold-in-a-given-amount-of-time}{%
\subsection{Can we predict whether an endangered species will pass a
certain population threshold in a given amount of
time?}\label{can-we-predict-whether-an-endangered-species-will-pass-a-certain-population-threshold-in-a-given-amount-of-time}}

We live in a time of decreasing bio-diversity and increasing numbers of
endangered species. When taking conservational measures with a species,
it is important to make predictions about the future of the species.
Existing metrics for this exist - notably, the MVP, or minimum viable
population. The minimum viable population (specifically the MVPa metric)
refers to, ``\ldots{}the minimum viable adult population size as
estimated from the available data regardless of study length,''
according to the article: ``Estimates of minimum viable population sizes
for vertebrates and factors influencing those estimates'' (cited at the
end of the essay).

What our model aims to do is predict for a given population of a species
with a known population growth rate and MVPa value whether they will
reach their MVPa threshold within 20 years. Essentially, can we expect
that the population will reach a minimum threshold for survivability
within the next 20 years?

Our model will produce a graph that plots aforementioned probability
versus the given population. To determine the parameters used for our
model, we will examine the whooping crane (\emph{grus americana}). The
parameters are: 1. Population growth rate: This is determined from the
population graph of the whooping crane obtained from the Center for
Biological Diversity. We used the population of 150 from 1995 and the
population of 450 from 2015 to obtain a grwoth of 300 cranes over the
last 20 years. Fit to an exponential model (which the population graph
follows), the year-over-year growth rate was approximately 5.65\%:

    \begin{enumerate}
\def\labelenumi{\arabic{enumi}.}
\setcounter{enumi}{1}
\tightlist
\item
  MVPa value: The following article lists the MVPa value for the
  whooping crane to be 1131 cranes: ``Estimates of minimum viable
  population sizes for vertebrates and factors influencing those
  estimates.'' This will be used as the threshold for success in our
  model. The final output will be a probability, and that proability is
  determined by the number times that the simulation produced a
  population that surpassed the MPVa value for the whooping crane.
\end{enumerate}

We will store these parameters as follows:

    \begin{Verbatim}[commandchars=\\\{\}]
{\color{incolor}In [{\color{incolor}29}]:} \PY{n}{system} \PY{o}{=} \PY{n}{System}\PY{p}{(}\PY{n}{t\PYZus{}0} \PY{o}{=} \PY{l+m+mi}{0}\PY{p}{,}
                         \PY{n}{t\PYZus{}final} \PY{o}{=} \PY{l+m+mi}{20}\PY{p}{,}
                         \PY{n}{growth\PYZus{}rate} \PY{o}{=} \PY{l+m+mf}{0.0565}
                         \PY{p}{)}
         
         \PY{n}{state} \PY{o}{=} \PY{n}{State}\PY{p}{(}\PY{n}{pop} \PY{o}{=} \PY{l+m+mi}{0}\PY{p}{,} \PY{n}{succ} \PY{o}{=} \PY{l+m+mi}{0}\PY{p}{,} \PY{n}{p\PYZus{}succ} \PY{o}{=} \PY{l+m+mi}{0}\PY{p}{,} \PY{n}{pop\PYZus{}20} \PY{o}{=} \PY{l+m+mi}{0}\PY{p}{,} \PY{n}{mvpa} \PY{o}{=}  \PY{l+m+mi}{1131}\PY{p}{)}
\end{Verbatim}


\begin{Verbatim}[commandchars=\\\{\}]
{\color{outcolor}Out[{\color{outcolor}29}]:} pop          0
         succ         0
         p\_succ       0
         pop\_20       0
         mvpa      1131
         dtype: int64
\end{Verbatim}
            
    Here we establish \texttt{run\_simulation1} which simulates population
increase for the first year, and \texttt{run\_simulation\_2\_20} which
simulates the population increase for the next twenty years. These are
separate functions so the first one can set the population value (which
is being swept) back to the starting value and the fucntion doesn't
compound itself when being run.

    \begin{Verbatim}[commandchars=\\\{\}]
{\color{incolor}In [{\color{incolor}30}]:} \PY{k}{def} \PY{n+nf}{run\PYZus{}simulation1}\PY{p}{(}\PY{n}{state}\PY{p}{,} \PY{n}{system}\PY{p}{,} \PY{n}{init\PYZus{}pop}\PY{p}{)}\PY{p}{:}
            \PY{c+c1}{\PYZsh{}this function is for computing the population after the first year            }
             
             \PY{n}{state}\PY{o}{.}\PY{n}{pop} \PY{o}{=} \PY{n+nb}{int}\PY{p}{(}\PY{n}{init\PYZus{}pop}\PY{p}{)}
             
             \PY{k}{for} \PY{n}{i} \PY{o+ow}{in} \PY{n+nb}{range} \PY{p}{(}\PY{n}{state}\PY{o}{.}\PY{n}{pop}\PY{p}{)}\PY{p}{:}
                
                 \PY{k}{if} \PY{n}{flip}\PY{p}{(}\PY{n}{system}\PY{o}{.}\PY{n}{growth\PYZus{}rate}\PY{p}{)}\PY{p}{:}
                     \PY{n}{state}\PY{o}{.}\PY{n}{pop} \PY{o}{+}\PY{o}{=} \PY{l+m+mi}{1}
             
             \PY{k}{return} \PY{n}{state}\PY{o}{.}\PY{n}{pop}
         
         \PY{k}{def} \PY{n+nf}{run\PYZus{}simulation2\PYZus{}20}\PY{p}{(}\PY{n}{state}\PY{p}{,} \PY{n}{system}\PY{p}{)}\PY{p}{:}
             \PY{c+c1}{\PYZsh{}this function is for computing the population for years 2 \PYZhy{} 20}
             
             \PY{k}{for} \PY{n}{i} \PY{o+ow}{in} \PY{n+nb}{range} \PY{p}{(}\PY{n}{state}\PY{o}{.}\PY{n}{pop}\PY{p}{)}\PY{p}{:}
                
                 \PY{k}{if} \PY{n}{flip}\PY{p}{(}\PY{n}{system}\PY{o}{.}\PY{n}{growth\PYZus{}rate}\PY{p}{)}\PY{p}{:}
                     \PY{n}{state}\PY{o}{.}\PY{n}{pop} \PY{o}{+}\PY{o}{=} \PY{l+m+mi}{1}
             
             \PY{k}{return} \PY{n}{state}\PY{o}{.}\PY{n}{pop}
\end{Verbatim}


    \begin{Verbatim}[commandchars=\\\{\}]
{\color{incolor}In [{\color{incolor}31}]:} \PY{k}{def} \PY{n+nf}{run\PYZus{}simulation20}\PY{p}{(}\PY{n}{state}\PY{p}{,} \PY{n}{system}\PY{p}{,} \PY{n}{init\PYZus{}pop}\PY{p}{)}\PY{p}{:}
             
             \PY{c+c1}{\PYZsh{}this function is for computing the final population at the end of the twenty years}
             \PY{c+c1}{\PYZsh{}run\PYZus{}simulation20 runs the two prior functions such that an ending population is calculated from an initial population}
             
             \PY{k}{for} \PY{n}{i} \PY{o+ow}{in} \PY{n}{linrange}\PY{p}{(}\PY{l+m+mi}{1}\PY{p}{,}\PY{l+m+mi}{20}\PY{p}{,}\PY{l+m+mi}{1}\PY{p}{)}\PY{p}{:}
                 
                 \PY{k}{if} \PY{n}{i} \PY{o}{==} \PY{l+m+mi}{1}\PY{p}{:}
                     \PY{n}{state}\PY{o}{.}\PY{n}{pop\PYZus{}20} \PY{o}{=} \PY{n}{run\PYZus{}simulation1}\PY{p}{(}\PY{n}{state}\PY{p}{,} \PY{n}{system}\PY{p}{,} \PY{n}{init\PYZus{}pop}\PY{p}{)} 
                 \PY{k}{if} \PY{n}{i} \PY{o}{\PYZgt{}}\PY{o}{=} \PY{l+m+mi}{2}\PY{p}{:}
                     \PY{n}{state}\PY{o}{.}\PY{n}{pop\PYZus{}20} \PY{o}{=} \PY{n}{run\PYZus{}simulation2\PYZus{}20}\PY{p}{(}\PY{n}{state}\PY{p}{,} \PY{n}{system}\PY{p}{)}
             
             \PY{k}{return} \PY{n}{state}\PY{o}{.}\PY{n}{pop\PYZus{}20}
\end{Verbatim}


    \begin{Verbatim}[commandchars=\\\{\}]
{\color{incolor}In [{\color{incolor}33}]:} \PY{n}{record} \PY{o}{=} \PY{n}{TimeSeries}\PY{p}{(}\PY{p}{)}
         \PY{k}{def} \PY{n+nf}{run\PYZus{}many\PYZus{}times}\PY{p}{(}\PY{n}{state}\PY{p}{,} \PY{n}{system}\PY{p}{,} \PY{n}{record}\PY{p}{)}\PY{p}{:}
             
             \PY{c+c1}{\PYZsh{}runs the simulation 100 times (many) to generate a time series of data. }
             \PY{c+c1}{\PYZsh{}The output is a probabilty for the chance a population with cross the MVPa threshold after 20 years.}
             
             \PY{n}{state}\PY{o}{.}\PY{n}{succ} \PY{o}{=} \PY{l+m+mi}{0}
             
             \PY{k}{for} \PY{n}{i} \PY{o+ow}{in} \PY{n+nb}{range}\PY{p}{(}\PY{l+m+mi}{100}\PY{p}{)}\PY{p}{:}
                 \PY{n}{record}\PY{p}{[}\PY{n}{i}\PY{p}{]} \PY{o}{=} \PY{n}{run\PYZus{}simulation20}\PY{p}{(}\PY{n}{state}\PY{p}{,} \PY{n}{system}\PY{p}{,} \PY{n}{init\PYZus{}pop}\PY{p}{)}
             
                 \PY{k}{if} \PY{n}{record}\PY{p}{[}\PY{n}{i}\PY{p}{]} \PY{o}{\PYZgt{}}\PY{o}{=} \PY{n}{state}\PY{o}{.}\PY{n}{mvpa}\PY{p}{:}
                     \PY{n}{state}\PY{o}{.}\PY{n}{succ} \PY{o}{+}\PY{o}{=} \PY{l+m+mi}{1}
             
         
             \PY{n}{state}\PY{o}{.}\PY{n}{p\PYZus{}succ} \PY{o}{=} \PY{n}{state}\PY{o}{.}\PY{n}{succ}
             \PY{k}{return} \PY{n}{state}
\end{Verbatim}


    Finally, we will call the \texttt{run\_many\_times} in a for loop that
sweeps the initial population.

    \begin{Verbatim}[commandchars=\\\{\}]
{\color{incolor}In [{\color{incolor}35}]:} \PY{n}{init\PYZus{}pop\PYZus{}array} \PY{o}{=} \PY{n}{linrange}\PY{p}{(}\PY{l+m+mi}{300}\PY{p}{,} \PY{l+m+mi}{500}\PY{p}{,} \PY{l+m+mi}{1}\PY{p}{)}
         
         \PY{n}{final\PYZus{}results} \PY{o}{=} \PY{n}{SweepSeries}\PY{p}{(}\PY{p}{)}
         
         \PY{k}{for} \PY{n}{init\PYZus{}pop} \PY{o+ow}{in} \PY{n}{init\PYZus{}pop\PYZus{}array}\PY{p}{:}
             \PY{c+c1}{\PYZsh{}sweeps different starting populations and runs the simulation many times}
             
             \PY{n}{state} \PY{o}{=} \PY{n}{run\PYZus{}many\PYZus{}times}\PY{p}{(}\PY{n}{state}\PY{p}{,} \PY{n}{system}\PY{p}{,} \PY{n}{record}\PY{p}{)}
             \PY{n}{final\PYZus{}results}\PY{p}{[}\PY{n}{init\PYZus{}pop}\PY{p}{]} \PY{o}{=} \PY{n+nb}{float}\PY{p}{(}\PY{n}{state}\PY{o}{.}\PY{n}{p\PYZus{}succ}\PY{p}{)}
\end{Verbatim}


    \begin{Verbatim}[commandchars=\\\{\}]
{\color{incolor}In [{\color{incolor}47}]:} \PY{n}{plot}\PY{p}{(}\PY{n}{final\PYZus{}results}\PY{p}{)}
         \PY{c+c1}{\PYZsh{}plots results}
         \PY{n}{decorate}\PY{p}{(}\PY{n}{xlabel}\PY{o}{=}\PY{l+s+s1}{\PYZsq{}}\PY{l+s+s1}{Initial Population}\PY{l+s+s1}{\PYZsq{}}\PY{p}{,} 
                  \PY{n}{ylabel}\PY{o}{=}\PY{l+s+s1}{\PYZsq{}}\PY{l+s+s1}{Probablity of Success(}\PY{l+s+s1}{\PYZpc{}}\PY{l+s+s1}{)**}\PY{l+s+s1}{\PYZsq{}}\PY{p}{,}
                  \PY{n}{title}\PY{o}{=}\PY{l+s+s1}{\PYZsq{}}\PY{l+s+s1}{Probablity of Success** vs Initial Whooping Crane Population}\PY{l+s+s1}{\PYZsq{}}\PY{p}{)}
\end{Verbatim}


    \begin{center}
    \adjustimage{max size={0.9\linewidth}{0.9\paperheight}}{output_12_0.png}
    \end{center}
    { \hspace*{\fill} \\}
    
    **Probability of surpassing MVPa in the next twenty years

    Above is the final result that shows the relationship between the
probability that, given the year-over-year growth rate is held constant,
the population of whooping cranes will reach the MVPa threshold
population of 1131.

As we predicted, the probability formed a logistic-shaped curve. We can
conclude that if the population of whooping cranes was below 350, there
would be a near-0\% chance that it would surpass the MVPa threshold in
the next twenty years; conversely, if given population of whooping
cranes was above 450, then it would be expected that the population
would exceed the MVPa threshold in the next twenty years.

    Our model does not exclusively apply to Whooping Crane populations. If
one knows the MVPa threshold for a specific organism and can assume that
the yearly growth rate is constant, our model can be used. Note that we
can (and did) assume constant yearly growth rate because no
recent/notable conservation efforts were made to protect the population.

To account for non-constant growth rate, we could investigate accounting
for non-constant growth rate as a function of time.

    https://www.esasuccess.org/2016/index.html

Estimates of minimum viable population sizes for vertebrates and factors
influencing those estimates:
http://www.wec.ufl.edu/academics/courses/wis4554/WebUpdate/ReadingsWIS5555/PVA/Reed\%20et\%20al\%20PVA\%20estimates\%202003BiolCons113\_23.pdf

    \emph{Godspeed}


    % Add a bibliography block to the postdoc
    
    
    
    \end{document}
